\section{Einleitung}
Das Betriebssystem UNIX, welches in den 1970er Jahren von AT\&T entwickelt wurde, wird als Fundament moderner Betriebssysteme angesehen. Linux und macOS werden
bis heute von Prinzipien, wie Portabilität, Modularität und dem Konzept \glqq Everything is a File\grqq \ geprägt \cite{ArtOfUnixProgramming, ModernOS}. Windows,
welches damals unabhängig von UNIX entwickelt wurde, beinhaltet heute mit dem \glqq Windows Subsystem for Linux\grqq \ (WSL) Merkmale von UNIX \cite{WSL}.
Diese historischen Wurzeln sind nicht nur in der Informatik spürbar, sondern wirken bis in Unternehmen, Behörden und den Alltag vieler Nutzerinnen und Nutzer
hinein, da zahlreiche digitale Geräte auf UNIX-Derivaten basieren \cite{FreeBSDOS, OSConcept}.

Der gegenwärtige Stand der Forschung macht deutlich, dass diese Gemeinsamkeiten und Unterschiede auf unterschiedlichen Ebenen existieren.
Silberschatz \cite{OSConcept} behandelt grundlegende Prozesse und Dateisysteme, während Raymond \cite{ArtOfUnixProgramming} den kulturellen Hintergrund der
UNIX-Philosophie betont.  McKusick \cite{FreeBSDOS} demonstriert die Weiterentwicklung dieser Konzepte in FreeBSD, während Tanenbaum \cite{ModernOS} aufzeigt,
dass auch andere Systeme wie Windows UNIX-ähnliche Funktionen integrieren.

Ausgehend von diesen Erkenntnissen lässt sich die zentrale Forschungsfrage folgendermassen formulieren:\\
\textit{In welchen Bereichen beeinflussen UNIX-Prinzipien moderne Betriebssysteme wie Linux, macOS und Windows und wo bestehen fundamentale Unterschiede?}

Ziel dieser Arbeit ist es, einem sowohl technisch versierten als auch allgemein interessierten Publikum aufzuzeigen, welche Konzepte moderner Betriebssysteme
direkt oder indirekt auf UNIX zurückgehen. Durch die Beantwortung der Forschungsfrage wird deutlich, in welchen Bereichen UNIX-Prinzipien bis heute relevant sind,
wo sich im Lauf der Zeit abweichende Lösungen etabliert haben und welche praktischen Folgen sich daraus für Nutzerinnen, Nutzer und Systemarchitekturen ergeben.
Die Untersuchung soll helfen, den Stellenwert dieser Prinzipien in aktuellen Systemen einzuordnen und ein besseres Verständnis für ihre Weiterentwicklung und
heutige Bedeutung zu schaffen.

Die Methoden, die ich verwende, bestehen hauptsächlich aus einer Literaturrecherche zu UNIX und modernen Betriebssystemen. Ergänzend führe ich ein Interview mit
einem Experten durch, um die theoretischen Erkenntnisse mit praktischen Erfahrungen zu verbinden.