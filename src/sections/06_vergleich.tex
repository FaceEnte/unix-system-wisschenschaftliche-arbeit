\section{Vergleich der UNIX-Anteile in modernen Betriebssysteme}

Diese Kapitel untersucht die Gemeinsamkeiten und Unterschiede moderner Betriebssysteme in Bezug auf ihre UNIX-Bestandteile. Dabei liegt der Fokkus auf der
Systemarchitektur, der Entwicklungsmethodik, sowie der praktischen Relevanz für Entwicklerinnen und Entwickler. Grundlage für diese Analyse bilden Fachliteraturen
und ein Interview mit Herrn von Niederhäusern \cite{interviewNH}.


\subsection{Gemeinsamkeiten und Unterschiede}

Obwohl Linux, macOS und Windows unterschiedliche Wurzen haben, lassen sich insbesondere zwischen Linux und macOS viele Gemeinsamkeiten feststellen, die auf ihr
UNIX-Erbe zurückzuführen sind. Beide Systeme unterstützen POSIX-Standards, verfügen über eine Shell-Umgebung, bieten weitgehende Kompatibilität mit UNIX-Tools und
setzen auf das Prinzip \glqq Everything is a File\grqq \ \cite{ArtOfUnixProgramming, ModernOS, OSConcept}.

macOS besitzt als zertifiziertes UNIX-System die enste technische Verbindung zur UNIX-Philosophie \cite{FreeBSDOS}. Es basiert auf BSD, einem klassichen
UNIX-Derivate, und implementiert standardisierte UNIX-Schnittstellen vollständig. Linux hingegen ist nicht zertifiziert, erfüllt aber in der Praxis viele der
gleichen Anforderungen und ist hochradig kompatibel mit UNIX-Anwendungen \cite{ModernOS, OSConcept}.

Windows stellt hingegen eine Sonderrolle dar, es ist weder UNIX-basiert noch vollständig POSIX-konform. Trotzdem hat Microsoft viele Elemente aus der UNIX-Welt
übernommen, beispielsweise mit dem Windows Subsystem for Linux, das es ermöglicht, native Linux Binaries unter Windows auszuführen \cite{WSL}. Auch Werkzeuge
wie Git Bash, Windows Terminal und die PowerShell zeigen die Annäherung an UNIX-Konzepte. Dennoch bleibt das zugrunde liegende Systemdesing stark unterschiedlich,
etwa in der Nutzung der Registry statt Konfigurationsdateien oder dem Fehlen eines nativen Paketmanagers \cite{interviewNH}.

Herr von Niederhäusern beschreibt die Situation folgendermassen: %! Zitat


\subsection{Einfluss auf Systemarchitekturen und Softwareentwicklung}

Die UNIX-Philosophie hat die Art, wie Betriebssysteme konzipiert und Software entwickelt wird, massgeblich geprägt. Zentral sind Konzepte wie Modularität,
textbasierte Schnittstellen, Skriptbarkeit und Wiederverwendbarkeit von Komponenten \cite{ArtOfUnixProgramming}.

Bei Linux und macOS ist dieser Einfluss direkt sichtbar, beide Systeme setzten auf klare Trennung von Benutzer- und Kernel-Modus, auf einfache, kombinierbare Tools,
sowie eine klare Prozesshierarchie \cite{ModernOS, FreeBSDOS}. Die Verwendung von Shell-Skripten, cronjobs und systemd (bzw. launchd bei macOS) illustrieren diese
Prinzipien.

Windows verfolgt traditionell einen stärker integrierten, GUI zentrierten Ansatz. Die Entwicklung der PowerShell und die Integration von WSL zeigen jedoch, dass
auch hier zunehmend UNIX-artige Denkweisen einfliessen \cite{WSL, OSConcept}. Für die Softwareentwicklung bedeutet das eine grössere Vereinheitlichung über
Plattformen hinweg. Entwickler nutzen heute plattformübergreifend Werkzeuge wie Git, Docker oder VSCode, alle stakr beeinflusst von UNIX-Standards.

Auch Herr von Niederhäusern unterstreicht diesen Wandel: \glqq Die Trennung der Welten ist vorbei. Ich sehe Entwickler, die unter Windows mit WSL und Docker
arbeiten und andere die auf dem Mac native UNIX-Tools einsetzten. UNIX ist nicht ott, es ist überall, in verschiedenen Gewändern\grqq \ \cite{interviewNH}.

Zusammenfassend lässt sich sagen, dass UNIX-Anteile heute die Basis für viele essentielle Konzepte in moderner Betriebssystemen bilden, ob direkt integriert wie bei
macOS, nachgebildet wie bei Linux oder teilweise emuliert wie bei Windows.