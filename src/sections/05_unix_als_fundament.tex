\section{UNIX als Fundament moderner Betriebssysteme}

UNIX ist mehr als nur ein historisches Betriebssystem, es bildet die Grundlage vieler moderner Systeme und beeinflusst bis heute deren Architektur, Konzepte und
Philosophie. In diesem Kapitel werden drei bedeutende Betriebssysteme betrachtetet, nämlich Linux, macOS und Windows, mit Fokus auf ihren Bezug zu UNIX.
Dabei werden sowohl Gemeinsamkeiten als auch Unterschiede analysiert. Neben wissenschaftlicher Literatur fliessen Aussagen aus dem Interview mit Herrn
von Niederhäusern ein, der als Systemadministrator und Dozent für das Modul \glqq Computing Infrastructure\grqq \ tätig ist.


\subsection{Linux: Ein UNIX-like System}

Linux ist kein direktes Derivat von UNIX, sondern ein sogenanntes UNIX-like System. Es wurde Anfang der 1990er Jahren von Linus Torvalds als freies Betriebssystem
entwickelt und orientiert sich stark an den Prinzipien und der Struktur klassischer UNIX-Systeme wie BSD und System V. \cite{ModernOS}

Wie Tanenbaum \& Bos beschrieben, übernimmt Linux zentrale UNIX-Konzepte wie das Dateisystem Hierarchiemodell, die Rechteverwaltung, die Interprozesskommunikation
(IPC) und di eNutzung von Textdateien zu Systemkonfiguration \cite{ModernOS}. Auch das Prinzip \glqq Everything is a File\grqq \ gilt nahezu uneingeschränkt
\cite{ArtOfUnixProgramming}.

Herr von Niederhäusern betonte: %! Zitat

Ein grosser Unterschied liegt in der Lizenzierung. Während klassische UNIX-Systeme proprietär waren, ist Linux unter der GNU General Public License (GPL)
frei verfügbar und wird gemeinschaftlich weiterentwickelt. Dennoch wurde Linux so konzipiert, dass es weitgehend POSIX-kompatibel ist, ein UNIX-Standard zur
Probabilität von Anwendungen \cite{OSConcept}.


\subsection{macOS: Ein UNIX-System}

macOS ist ein zertifiziertes UNIX-System (seit Version 10.5 \glqq Leopard\grqq), das auf dem Darwin-Kernel basiert. Darwin wiederum vereint Komponenten von BSD,
insbesondere FreeBDS, mit dem mach-Mikrokernel \cite{FreeBSDOS}. Dadurch ist macOS technisch gesehen näher an UNIX als Linux.

Laut Apple erfüllt macOS die Anforderungen der Single UNIX Specification (SUS), wodurch es sich offiziell als UNIX-System bezeichnen darf. Für Entwickler bedeutet
das viele auf UNIX basierende Werkzeuge und Skripte nativ funktionieren, was macOS zu einer beliebten Plattform für Softwareentwicklung macht \cite{OSConcept}.

Herr von Niederhäusern erklärte dazu: %! Zitat

macOS integriert die UNIX-Basis in eine moderne, grafische Benutzeroberfläche. Dies zeigt, wie UNIX-Konzepte auch in einem kommerziellen Mainstream-System überleben
und weiterentwickelt werden können \cite{FreeBSDOS}.


\newpage
\subsection{Windows: UNIX-Elemente in einem nicht-UNIX-System}

Microsoft Windows basiert historisch nicht auf UNIX, sondern entwickelte sich aus MS-DOS. Dennoch hat WWindows im Laufe der Zeit zahlreiche UNIX-ähnliche Elemente
übernommen, primär im Server- und Entwicklerbereich \cite{ModernOS}.

Ein Beispiel ist das Windows Subsystem for Linux (WSL), das seit Windows 10 verfügbar ist. Es erlaubt die native Ausführung von Linux Distributionen unter Windows,
inklusive Zugriff auf Bash, SSH, Git und andere UNIX-Tools \cite{WSL}. Das macht Windows für Entwickler attraktiver, die bisher UNIX-Umgebungen bevorzugt haben.

Zudem implementiert Windows zunehmend POSIX-nahe Schnittstellen, allerdings nicht vollständig. Herr von Niederhäusern wies darauf hin: \glqq Windows hat viele gute
Tools aus der UNIX-Welt kopiert oder emuliert, aber es fühlt sich nie ganz gleich an\grqq \cite{interviewNH}.

Auch Systemdienste wie Prozess- und Speicherverwaltung folgen einem anderen Design. Dennoch zeigen Entwicklungen wie Windows Terminal oder native OpenSSH
Unterstützung die zunehmende Annäherung an UNIX-Paradigmen, getrieben von den Anforderungen moderner Softwareentwicklung und Cloud-Infrastrukturen
\cite{ArtOfUnixProgramming, OSConcept}.