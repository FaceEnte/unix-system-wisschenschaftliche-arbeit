%! Author = byteme
%! Date = 13.05.2025

% Document
\section{Abstract}

Die vorliegende Arbeit befasst sich mit der Relevanz und dem Einfluss des UNIX-Betriebssystems auf moderne Betriebssysteme. UNIX wurde in den 1970er Jahren
entwickelt und zeichnet sich durch Prinzipien wie Modularität, Portabilität, einfache Schnittstellen sowie das Paradigma \glqq Everything is a File\grqq \ aus
\cite{ArtOfUnixProgramming, ModernOS}. Diese Eigenschaften haben sich als so grundlegend erwiesen, dass sie in vielen heute verbreiteten
Betriebssystemen weiterhin erkennbar sind, etwa in direkter Form wie bei macOS, in adaptierter Weise wie bei Linux oder als einzelne Elemente in ursprünglich
UNIX-fremden Systemen wie Microsoft Windows \cite{ModernOS, WSL, OSConcept}. Ziel der Arbeit ist es, aufzuzeigen, in welchen Bereichen UNIX-Prinzipien bis
heute technologische und konzeptionelle Wirkung entfalten und wo sich im Laufe der Zeit alternative Ansätze durchgesetzt haben.

Die Untersuchung kombiniert theoretische Analyse mit praxisnaher Perspektive. Einerseits wurde umfangreiche Fachliteratur zu UNIX,
Betriebssystemarchitekturen und modernen Systemen ausgewertet, darunter Werke von Raymond, Tanenbaum, Silberschatz und McKusick
\cite{ArtOfUnixProgramming, ModernOS, FreeBSDOS, OSConcept}.

Andererseits wurde ein Experteninterview mit Herrn von Niederhäusern geführt, einem erfahrenen Systemadministrator und Linux Nutzer, um theoretische Erkenntnisse
mit praktischen Einschätzungen zu ergänzen \cite{interviewNH}. Der methodische Ansatz vergleicht Linux, macOS und Windows entlang von Kriterien wie Modularität,
POSIX-Kompatibilität, Kommandozeilenunterstützung und Toolphilosophie.

Die Analyse zeigt, dass Linux als UNIX-like System viele klassische UNIX-Konzepte übernommen und weiterentwickelt hat. Es ist zwar kein direktes
Derivat, setzt jedoch auf zentrale Prinzipien wie Dateisystemdesign, Nutzerverwaltung und Interprozesskommunikation \cite{ModernOS, OSConcept}. macOS ist ein
zertifiziertes UNIX-System, basiert auf dem BSD Zweig und Mach-Kernel und kombiniert technische Nähe zu UNIX mit moderner Benutzerfreundlichkeit
\cite{FreeBSDOS, OSConcept}. Windows hingegen nutzt mit dem Windows Subsystem for Linux eine Emulationsschicht, um UNIX-Funktionalität bereitzustellen
\cite{WSL, interviewNH}.

Gleichzeitig zeigen sich deutliche Unterschiede. Besonders Windows unterscheidet sich durch seine Architektur und geringere CLI-Orientierung. Dennoch ist ein
klarer Trend zur Modularisierung und Automatisierung erkennbar, wie sie im UNIX-Umfeld etabliert sind \cite{ArtOfUnixProgramming, interviewNH}. Diese Prinzipien
gewinnen insbesondere in Cloud Computing, DevOps und Containerisierung an Bedeutung, wo UNIX-nahe Systeme wie Linux dominieren
\cite{ModernOS, OSConcept, interviewNH}.

Die Ergebnisse verdeutlichen, dass UNIX bis heute die Grundlage vieler zentraler IT-Konzepte bildet. In Serverlandschaften, Softwareentwicklung, Cloud- und
Edge-Infrastrukturen bleiben UNIX-basierte Systeme prägend. Auch zukünftige Entwicklungen wie Containerisierung oder maschinelles Lernen bauen auf modularen,
skalierbaren Konzepten auf \cite{ArtOfUnixProgramming, ModernOS, interviewNH}. Der im Interview betonte hohe Stellenwert der Modularität bestätigt diese Tendenz
\cite{interviewNH}.

Abschliessend lässt sich festhalten, dass UNIX weiterhin eine tragende Rolle in der Informatik spielt. Ob durch direkte Nachfolger wie macOS, offene Systeme wie
Linux oder hybride Ansätze wie WSL, die UNIX-Philosophie lebt in vielfältiger Form weiter und bleibt technologisch wie konzeptionell hochrelevant
\cite{ArtOfUnixProgramming, interviewNH}.